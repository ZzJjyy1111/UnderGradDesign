% !TEX root = ../main.tex

\chapter{连续时间系统下多刚体系统姿态一致性}
\section{引言}
\section{问题描述}
考虑$n$个刚体组成的多刚体系统,每个刚体的姿态动力学服从\ref{eq:dynamic1}\ref{eq:dynamic2},即
\begin{equation}
    \label{eq:dynamic3}
    \dot R_i=R_i\omega_i^\wedge
\end{equation}
\begin{equation}
    J_i\dot\omega_i=-\omega_i^\wedge J_i\omega_i+\tau_I
\end{equation}
其中$R_i\in\text{SO(3)}$是刚体系相对于惯性系的旋转矩阵,$\omega_i\in\mathbb{R}^3$为刚体角速度,
$J_i\in\mathbb{R}^{3\times 3}$为刚体转动惯量矩阵,$\tau_i\in\mathbb{R}^3$为加在刚体上的力矩,也就是控制算法需要给出的控制律。
$\omega_i$、$J_i$、$\tau_i$都是在刚体系中给出的,而非惯性系下的。

在此多刚体系统中,我们假设每个刚体$i$能通过测量获得它与部分其他刚体$j$($a_{ij}>0$)之间的相对信息,
并且还能通过自身配备的传感器获得自身绝对角速度信息$\omega_i$。
在本章第三节有相对角速度测量的情况下,相对信息指的是相对姿态$R_j^TR_i$和相对角速度$\omega_i-R_i^TR_j\omega_j$。
在本章第四届无相对角速度测量的情况下,相对信息指的是仅相对姿态$R_j^TR_i$。

基于以上信息,我们希望为此多刚体系统设计一个协议,使得各刚体姿态达到一致,即$R_i=R_j$,$\forall i,j\in\mathcal{V},i\not=j$。

\section{有相对角速度测量}
此部分相对简单,已有一些研究人员从其他角度出发进行了研究\cite{maadani20226}\cite{sarlette2009autonomous},
但为了本文内容的完整性,我们仍在此处给出控制协议并作分析证明。
为解决上述多刚体系统姿态一致性问题,我们给出以下控制协议:
\begin{equation}
    \label{eq:protocol1}
    \tau_i=-k_1\sum_{j=1}^na_{ij}(R_j^TR_i-R_i^TR_j)^{\vee}-k_2\sum_{j=1}^na_{ij}(\omega_i-R_i^TR_j\omega_j)-k_3\omega_i
\end{equation}
其中$k_1$,$k_2$,$k_3$是三个大于0的常数。

\subsection{一致性分析}
利用Lyapunov稳定性理论,分析多刚体系统\ref{eq:dynamic3}在控制协议\ref{eq:protocol1}下的一致性,得到以下结果。
\begin{theorem}
    考虑多刚体系统\ref{eq:dynamic3}和控制协议\ref{eq:protocol1}。若通讯拓扑图$\mathcal{G}$是一个无向连通图,则
    $\lim_{t\rightarrow\infty}\omega_i=0$,$\lim_{t\rightarrow\infty}\sum_{j=1}^na_{ij}(R_j^TR_i-R_i^TR_j)^\vee=0,\forall i$。
    若$\mathcal{G}$是一棵无向树,则多刚体系统姿态能达到一致,即$R_i=R_j$。
    \begin{proof}
        
        首先计算$\text{tr}(R_j^TR_i)$对时间的导数:
        \begin{equation}
            \begin{aligned}
                \dot{(R_j^TR_i)}&=(R_j\omega_j^\wedge)^TR_i+R_j^TR_i\omega_i^\wedge\\
                &=R_j^TR_i(-R_i^TR_j\omega_j^\wedge R_j^TR_i+\omega_i^\wedge)\\
                &=R_j^TR_i(\omega_i-R_i^TR_j\omega_j)^\wedge\\
                \dot{\text{tr}(R_j^TR_i)}&=\text{tr}(\dot{R_j^TR_i})\\
                &=-(\omega_i-R_i^TR_j\omega_j)^T(R_j^TR_i-R_i^TR_j)^\vee
            \end{aligned}
        \end{equation}
        以上用到了\ref{lm:trace}和\ref{eq:fanduichen}。

        设李雅普诺夫函数为
        \begin{equation}
            V=\frac{k_1}{2}\sum_{i=1}^n \sum_{j=1}^n a_{ij}\text{tr}(I_3-R_j^TR_i)+\frac{1}{2}\sum_{i=1}^n\omega_i^TJ_i\omega_i
        \end{equation}
        将$V$对时间求导得到
        \begin{equation}
            \dot V=\sum_{i=1}^n\sum_{j=1}^n\frac{k_1}{2}a_{ij}(\omega_i-R_i^TR_j\omega_j)^T(R_j^TR_i-R_i^TR_j)^\vee+\sum_{i=1}^n\omega_i^TJ_i\dot\omega_i
        \end{equation}
        其中第一项可以写成
        \begin{equation}
            \dot V_1=\sum_{i=1}^n\sum_{j=1}^n\frac{k_1}{2}a_{ij}\omega_i^T(R_j^TR_i-R_i^TR_j)^\vee-\sum_{i=1}^n\sum_{j=1}^n\frac{k_1}{2}a_{ij}(R_i^TR_j\omega_j)^T(R_j^TR_i-R_i^TR_j)^\vee
        \end{equation}

        由\ref{lm:trace},对于旋转矩阵$A\in\text{SO(3)}$和三维列向量$x$,有
        \begin{equation}
            (Ax)^T(A^T-A)^\vee=\text{tr}((Ax)^\wedge A)=\text{tr}(Ax^\wedge A^TA)=\text{tr}(Ax^\wedge)=-x^T(A-A^T)
        \end{equation}
        因此将$\dot V_1$进一步化简得到
        \begin{equation}
            \dot V_1=\sum_{i=1}^n\sum_{j=1}^n\frac{k_1}{2}a_{ij}\omega_i^T(R_j^TR_i-R_i^TR_j)^\vee+\sum_{i=1}^n\sum_{j=1}^n\frac{k_1}{2}a_{ij}\omega_j^T(R_i^TR_j-R_j^TR_i)^\vee
        \end{equation}
        
        由于拓扑图为无向图,$a_{ij}=a_{ji}$,故
        \begin{equation}
            \begin{aligned}
                \dot V_1&=\sum_{i=1}^n\sum_{j=1}^n\frac{k_1}{2}a_{ij}\omega_i^T(R_j^TR_i-R_i^TR_j)^\vee+\sum_{i=1}^n\sum_{j=1}^n\frac{k_1}{2}a_{ji}\omega_j^T(R_i^TR_j-R_j^TR_i)^\vee\\
                &=\sum_{i=1}^n\sum_{j=1}^n\frac{k_1}{2}a_{ij}\omega_i^T(R_j^TR_i-R_i^TR_j)^\vee+\sum_{j=1}^n\sum_{i=1}^n\frac{k_1}{2}a_{ji}\omega_j^T(R_i^TR_j-R_j^TR_i)^\vee\\
                &=\sum_{i=1}^n\sum_{j=1}^nk_1a_{ij}\omega_i^T(R_j^TR_i-R_i^TR_j)^\vee
                \end{aligned}
        \end{equation}
        
        将$\dot V_1$与控制协议\ref{eq:protocol1}代入得到
        \begin{equation}
            \begin{aligned}
                \dot V&=\sum_{i=1}^n\sum_{j=1}^nk_1a_{ij}\omega_i^T(R_j^TR_i-R_i^TR_j)^\vee+\sum_{i=1}^n\omega_i^T(-\omega_i\times J_i\omega_i+\tau_i)\\
                &=\sum_{i=1}^n\sum_{j=1}^nk_1a_{ij}\omega_i^T(R_j^TR_i-R_i^TR_j)^\vee+\sum_{i=1}^n\omega_i^T(-k_1\sum_{j=1}^na_{ij}(R_j^TR_i-R_i^TR_j)^{\vee}\\
                &-k_2\sum_{j=1}^na_{ij}(\omega_i-R_i^TR_j\omega_j)-k_3\omega_i)\\
                &=-k_2\sum_{i=1}^n\sum_{j=1}^n\omega_i^T(\omega_i-R_i^TR_j\omega_j)-\sum_{i=1}^nk_3\omega_i^T\omega_i\\
                &=k_2\sum_{i=1}^n\sum_{j=1}^n(R_i\omega_i)^T(R_i\omega_i-R_j\omega_j)-\sum_{i=1}^nk_3\omega_i^T\omega_i
                \end{aligned}
        \end{equation}
        其中用到了$\omega_i^T(\omega^\wedge J_i\omega_i)=0$。令$R_i\omega_i=\Omega_i$,$\Omega=[\Omega_1^T,\Omega_2^T,...,\Omega_n^T]^T$
        \begin{equation}
            \dot V=-k_2\Omega^T(\mathcal{L}\otimes I_3)\Omega-\sum_{i=1}^nk_3\omega_i^T\omega_i\leq0
        \end{equation}

        令不变集为$\{R_j^TR_i-I_3,\omega_i|\dot V=0\}$。当$\dot V\equiv0$时,$\omega_i\equiv0$,
        由动力学方程\ref{eq:dynamic3}可以得到$\tau_i\equiv0$,代入控制协议\ref{eq:protocol1}得
        \begin{equation}
            \label{eq:equillibrium}
            \sum_{j=1}^na_{ij}(R_j^TR_i-R_i^TR_j)^\vee=0,\quad \forall i
        \end{equation}

        当通讯拓扑图$\mathcal{G}$是一棵无向树时,边集$\mathcal{E}$中的元素个数为$n-1$。
        不妨令$i<j$,令$\hat q\in\mathbb{R}^{3(n-1)}$为所有向量$(R_j^TR_i-R_i^TR_j)^\vee$的组合,$(i,j)\in\mathcal{E}$。
        注意到$(R_j^TR_i-R_i^TR_j)^\vee=-(R_i^TR_j-R_j^TR_i)^\vee$,\ref{eq:equillibrium}可以写成
        \begin{equation}
            (Q\otimes I_3)\hat q=0
        \end{equation}
        其中$Q\in\mathbb{R}^{n\times(n-1)}$,是通讯拓扑图$\mathcal{G}$的关联矩阵。
        根据引理\ref{lm:incidence},$\mathcal{G}$是一棵无向树,矩阵$Q$满秩。
        因此$\hat q=0$,即对相邻的$i,j$,$(R_j^TR_i-R_i^TR_j)^\vee=0$,
        根据引理\ref{lm:trace}和图$\mathcal{G}$的连通性,$R_i=R_j,\forall i\not=j$,姿态达成一致。

        证毕。
    \end{proof}
\end{theorem}

此处姿态一致性的达成有赖于图$\mathcal{G}$是一棵树这一特殊性质,仅依靠$\sum_{j=1}^na_{ij}(R_j^TR_i-R_i^TR_j)^\vee=0,\quad \forall i$这一结果无法得出一致性的结论。
事实上,从仿真结果来看,图$\mathcal{G}$是否是树并不影响一致性,只要图是连通的,姿态一致性都能达成。
关于这一问题,将在本章第六部分进行讨论分析。


\section{无相对角速度测量}
\section{仿真验证}
\section{关于平衡点的讨论}
\section{本章小结}