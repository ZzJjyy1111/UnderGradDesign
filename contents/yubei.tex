% !TEX root = ../main.tex

\chapter{预备知识}
多刚体系统是一种特殊的多智能体系统,因此在本文接下来的分析计算中,不仅需要用到多智能体系统研究中常用的图论知识、矩阵知识,还需要用到刚体姿态动力学的相关知识。
\section{图论知识}
多刚体系统之间的信息交互网络可以用拓扑图$\mathcal{G}=\{\mathcal{V},\mathcal{E}\}$描述\cite{ren2008distributed}。
$\mathcal{V}=\{1,2,...,n\}$是图的点集,代表系统中的各个刚体,其中$n$为刚体个数。
$\mathcal{E}\subset\mathcal{V}\times\mathcal{V}$是图的边集,其中每个元素(边)是点集中元素的有序对。
边$(i,j)\in\mathcal{E}$代表刚体$j$能获得刚体$i$的信息,此时$i$称为$j$的邻居,或者称$i$为父(首)节点,$j$为子(尾)节点。
点$i$的相邻节点集合为$\mathcal{N}_i=\{j\in\mathcal{V}|(j,i)\in\mathcal{E}\}$,代表所有能让刚体$i$接收到信息的其他刚体。
当$\mathcal{G}$为无向图时,若边$(i,j)\in\mathcal{E}$,则边$(j,i)\in\mathcal{E}$,也就是说刚体之间的通信是相互的,且两者互为邻居。
首尾相连但不形成环的边的集合称为路径,例如$\{(i_1,i_2),(i_2,i_3),...,(i_{m-1},i_m)\}$。
若无向图$\mathcal{G}$中两个点之间都有至少一条路径将它们相连,则称$\mathcal{G}$为连通图。
若无向图$\mathcal{G}$中的任意两个点之间有且仅有一条路径将它们相连,则把$\mathcal{G}$称为树。

除了用集合描述图,还能用矩阵刻画图的性质。图$\mathcal{G}$可以用邻接矩阵$\mathcal{A}$来描述,$\mathcal{A}=[a_{ij}]_{n\times n}$。
若边$(j,i)$存在边集中,即刚体$i$能收到刚体$j$的信息,则$a_{ij}>0$;若边$(j,i)$不存在边集中,即刚体$i$不能收到刚体$j$的信息,则$a_{ij}=0$。
$a_{ij}$的大小代表边的权重,对于平衡图,所有非零的$a_{ij}$设置为$1$,或其他相同的值。
对于节点$i$,所有从$i$出发的边的权重之和称为$i$的出度$d^i_{out}$,$d^i_{out}=\sum_{j=1}^na_{ji}$;
所有到达$i$的边的权重之和称为$i$的入度$d_{in}^i$,$d_{in}^i=\sum_{j=1}^na_{ij}$。
当$\mathcal{G}$为无向图时,邻接矩阵$\mathcal{A}$为对称矩阵,每个节点的入度等于出度。
图$\mathcal{G}$的入读矩阵定义为$\mathcal{D}=\text{diag}(d_{in}^1,d_{in}^2,...,d_{in}^n)$。
图$\mathcal{G}$的拉普拉斯矩阵定义为$\mathcal{L}=[l_{ij}]_{n\times n}=\mathcal{D}-\mathcal{A}$。或者写成
\begin{equation}
    l_{ii}=\sum_{j=1,j\not=i}^n a_{ij},\quad l_{ij}=-a_{ij},i\not=j
\end{equation}
关于$\mathcal{L}$和$x=[x_1^T,x_2^T,...,x_n^T]^T$(每个$x_i$都是$m$维度的向量)通常有以下等式:
\begin{equation}
    \sum{j=1}^na_{ij}(x_i-x_j)=(\mathcal{L}_i\otimes I_m)x
\end{equation}
\begin{equation}
    \sum_{i=1}^n\sum_{j=1}^na_{ij}x_i^T(x_i-x_j)=x^T(\mathcal{L}\otimes I_m)x
\end{equation}
其中$\mathcal{L}_i$表示矩阵$\mathcal{L}$的第$i$行。
定义图$\mathcal{G}$的关联矩阵为$Q=[q_{ik}]\in\mathbb{R}^{n\times p}$,
其中$p$为边集$\mathcal{E}$中元素的个数。
若某条边$e_k$从节点$i$出发到节点$j$终止,则$q_{ik}=1$,$q_{jk}=-1$,第$k$列的其余各元素为0。
关于关联矩阵$Q$,有引理如下
\begin{lemma}
    \label{lm:incidence}
    若无向图$\mathcal{G}$是一棵树,则关联矩阵$Q$满秩。
\end{lemma}
\section{矩阵知识}
\begin{definition}\cite{ren2008distributed}
    对于给定矩阵$A=[a_{ij}]\in\mathbb{R}^{p\times q}$,$B=[b_{ij}]\in\mathbb{R}^{m\times n}$,它们的Kronecker积为
    \begin{equation}
        A\otimes B=\left[\begin{matrix} a_{11}B&\cdots&a_{1q}B\\\vdots&\ddots&\vdots\\a_{p1}B&\cdots&a_{pq}B\end{matrix}\right]\in\mathbb{R}^{pm\times qn}
    \end{equation}
\end{definition}
\begin{lemma}
    对于相同维度的方阵$A$,$B$,$C$,关于矩阵的迹(trace),有如下事实
    \begin{equation}
        \text{tr}(AB)=\text{tr}(BA)
    \end{equation}
    \begin{equation}
        \text{tr}(ABC)=\text{tr}(BCA)=\text{tr}(CAB)
    \end{equation}
\end{lemma}
\begin{lemma}[杨氏不等式]
    \label{lm:young}
    对$\forall x,y\in\mathbb{R}$,有$\epsilon>0$,$xy<\frac{\epsilon}{2}x^2+\frac{1}{2\epsilon}y^2$。
    
    拓展到向量,对$\forall x,y\in\mathbb{R}^m$,有$\epsilon>0$,$x^Ty<\frac{\epsilon}{2}\parallel x\parallel^2+\frac{1}{2\epsilon}\parallel y\parallel^2$。
    其中$\parallel \cdot \parallel$表示向量的2范数 
\end{lemma}
\begin{lemma}\cite{ren2008distributed}
    \label{eigenL}
    对于有向图$\mathcal{G}$,对应拉普拉斯矩阵$\mathcal{L}$,图$\mathcal{G}$含有有向生成树,当且仅当0是$\mathcal{L}$的单特征值,且$\mathcal{L}$的其他特征值实部均大于0。
    对于无向图$\mathcal{G}$,对应拉普拉斯矩阵$\mathcal{L}$,图$\mathcal{G}$是联通的,当且仅当0是$\mathcal{L}$的单特征值。
\end{lemma}
\section{刚体知识}
刚体的姿态由刚体系$\mathcal{F_B}$相对惯性系$\mathcal{F_I}$的旋转描述,
通常用旋转矩阵$R\in \text{SO(3)}$表示。$\text{SO(3)}$是特殊正交群,$\text{SO(3)}=\{R\in \mathbb{R}^{3\times 3}|\det R =1,RR^T=R^TR=I_3\}$。
同时,旋转也可以用欧拉角$\theta\in(-\pi,\pi]$与欧拉轴$u\in \mathbb{S}^2$表示,这两个量代表刚体系相对惯性系绕$u$逆时针旋转了$\theta$。
旋转矩阵与这两者的关系由罗德里格斯旋转公式\cite{sola2018micro}给出:
\begin{equation}
    R=I+u^\wedge\sin\theta+(u^\wedge)^2(1-\cos\theta)
\end{equation}
也可以表示为指数形式
\begin{equation}
    R=\exp((\theta u)^\wedge)
\end{equation}
以上两式中,$(\cdot)^\wedge$将一个三维向量转换成一个三维方阵,对于向量$x=[x_1,x_2,x_3]^T\in\mathbb{R}^3$
\begin{equation}
    x^\wedge=\left[\begin{matrix} 0&-x_3&x_2\\x_3&0&-x_1\\-x_2&x_1&0\end{matrix}\right]
\end{equation}
与之互为逆运算的$(\cdot)^\vee$将一个对角线上元素全为0的三维方阵转换为一个三维向量
\begin{equation}
    \left(\left[\begin{matrix} 0&-x_3&x_2\\x_3&0&-x_1\\-x_2&x_1&0\end{matrix}\right]\right)^\vee=x=[x_1,x_2,x_3]^T
\end{equation}
值得注意的是,上述三维方阵为反对称矩阵,因此有
\begin{equation}
    \label{eq:fanduichen}
    (x^\wedge)^T=-x^\wedge
\end{equation}

关于旋转矩阵及$(\cdot)^\wedge$$(\cdot)^\vee$运算,先给出如下引理
\begin{lemma}\cite{zou2017rotation}
    \label{lm:trace}
    对于向量$x\in\mathbb{R}^3$ 和矩阵$R\in\text{SO(3)}$,$R$对应的欧拉角$\theta$,如下事实成立
    \begin{equation}
        \text{tr}((x^\wedge R))=-x^T(R-R^T)^\vee
    \end{equation}
    \begin{equation}
        (x^\wedge R+R^Tx^\wedge)^\vee=(\text{tr}(R)I_3-R)x
    \end{equation}
    \begin{equation}
        (Rx)^\wedge=Rx^\wedge R^T
    \end{equation}
    \begin{equation}
        \parallel\frac{(R-R^T)^\vee}{2}\parallel=\cos\frac{\theta}{2}\sqrt{\text{tr}(I_3-R)}
    \end{equation}
    \begin{equation}
        \text{tr}(I_3-R)=2(1-\cos\theta)\in[0,4)
    \end{equation}
    $(R-R^T)^\vee=0$,当且仅当$R=I_3$时成立。
\end{lemma}

对于单个刚体,其姿态动力学满足
\begin{equation}
    \label{eq:dynamic1}
    \dot R=R\omega^\wedge
\end{equation}
\begin{equation}
    \label{eq:dynamic2}
    J\dot \omega=-\omega^\wedge J\omega+\tau
\end{equation}
其中$R\in\text{SO(3)}$是刚体系相对于惯性系的旋转矩阵,$\omega\in\mathbb{R}^3$为刚体角速度,$J\in\mathbb{R}^{3\times 3}$为刚体转动惯量矩阵,$\tau\in\mathbb{R}^3$为加在刚体上的力矩,也就是控制算法需要给出的控制律。
$\omega$、$J$、$\tau$都是在刚体系中给出的,而非惯性系下的。
\section{其他引理}
\begin{lemma}\cite{lin2012new}
    令分段连续函数$\phi(t):\mathbb{R}_{\geq0}\rightarrow\mathbb{R}$满足以下条件

    条件1:$\phi(t)$在任意分段区间$(t_{r_j},t_{r_{j+1}})$内是连续与可导的,并在$t_{r_j}$处发生切换,且$\phi(t)$是在$t_{r_j}$处右连续,右可导,$r\in\mathbb{Z}_{\geq0}$,$j=0,1,...,m_r-1$。

    条件2:$\phi(t)$的导数(右导数)在$t\in[0,+\infty)$是有界的,即对于某个常数$\omega>0$,$|\dot\phi(t)|\leq\omega$在$t\in[0,+\infty)$恒成立。

    条件3:$\lim_{t\rightarrow+\infty}\int_{0}^t\phi(\tau)d\tau$存在且有限。

    可以得到
    \begin{equation}
        \lim_{t\rightarrow\infty}\phi(t)=0
    \end{equation}
\end{lemma}
\section{本章小结}
本章介绍了研究多刚体系统所需的相关数学及物理知识,包括图论知识、矩阵知识、刚体知识,引入了一些在后续证明中需要用到的引理定理。
